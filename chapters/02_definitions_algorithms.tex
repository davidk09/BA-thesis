!TeX root = ../main.tex
% Add the above to each chapter to make compiling the PDF easier in some editors.

\chapter{Definitions and Algorithms}\label{chapter:definitions_algorithms}

\section{Convex Hull}

\subsection{Basics}
\label{sec:basics}
First the Convex Hull will be defined. A set $ s \subseteq \mathbb{R}^2 $ is convex if for  every
two points p and q in $s$ it holds that all points on the line segment connecting p and q 
are in $s$ again. This can be expressed, as the fact that the any convex combination of 
p and q has to be in $s$ again. In Isabelle the convex predicate is defined exactly this way:
\begin{lstlisting}
  definition convex :: 'a real_vector set $\Rightarrow$ bool where
  convex s $\longleftrightarrow$ ($\forall$x$\in$s. $\forall$y$\in$s. $\forall$u$\geq$0. $\forall$v$\geq$s. u + v = 1 $\longrightarrow$ u *$_R$ x + v *$_R$ y $\in$ s)
\end{lstlisting}
The convex hull of a set $s$ is the smallest convex set in which $s$ is contained. 
There are several alternative ways in which the convex hull can be defined.
One possible way is to define it as the intersection of all convex sets containing $s$, which 
is also the definition used in Isabelle/HOL. We have already seen the convex predicate, the hull
predicate is defined as the intersection of all sets t that contain s and fulfill the predicate S.
\begin{lstlisting}
    definition hull :: (a' set $\Rightarrow$ bool) $\Rightarrow$ a' set $\Rightarrow$ a' set where
    S hull s = $\bigcap${t. S t $\land$ s $\subseteq$ t}
\end{lstlisting}
Consequently \lstinline|convex hull s| refers to the intersection of all convex sets that 
contain $s$ and therefore the convex hull of the set $s$.
In the two dimensional case for a finite $s \subset \mathbb{R}^2$, the convex hull CH of $s$ 
is a convex polygon and all the corners of this convex polygon are points from S (see figure 1). \cite{de2000computational}
As this thesis will focus on the two dimensional case and only give an outlook on the the 
three dimensional case, we will deal with computing the convex hull of $s \in \mathbb{R}^2$ 
in the following and therefore computing a convex polygon as representation of the convex hull of $s$.  
Assuming no three points in $s$ are colinear, then the edges $E \subseteq s^2$  of the polygon 
can be described as exactly those $(p,q) \in s^2$ for which all points in $s$ lie on the left of the vector $\vec{pq}$. 
Notice that the direction of the vector i.e. from p to q is relevant for expressing that a point
lies on the left of the vector $\vec{pq}$.
Of course the symmetric definition of E as those $(p,q) \in s^2$ for which all points in $s$ lie on the right of the line 
$\vec{pq}$ works as well. The only difference is that in the set of directed edges we get, 
every edge now points into the opposite direction. Both definitions make sense, but 
because there is already infrastructure in place for first definition i.e. $(p,q)$  is 
an edge if and only if all points in s are left of $\vec{pq}$, we will use this definition.
But first we need to state the concept of a point $q$ being left of the vector $\vec{pq}$ more
precicsely, especially when there can be three colinear points in $s$.

\subsection{Orientation}
Figure x shows the convex hull of the points $s = \{p_0,p_1,p_2,p_3\}$ in the form of a convex polygon.
When using the previous definition, $(p_1,p_2)$, $(p_2,p_3)$ and $(p_1,p_3)$
would be edges of the convex polygon, because it holds that all points in $s$ are left 
of $\vec{p_1 p_2}$, left of $\vec{p_2 p_3}$ and left of $\vec{p_1 p_3}$. 
This is an unintuitive definition which should be avoided. Therefore we define the 
condition for $(p,q)$ to be an edge of the convex hull polygon more precicsely.
$(p,q) \in s^2$ is an edge of the convex hull polygon if and only if all points $r \in s$ 
are either strictly left of the vector $\vec{p q}$ (p, q and r are not colinear) or 
r is contained in the closed segment between $p$ and $q$. The second part can be written
as \lstinline|r $\in$ closed_segment p q| in Isabelle where \lstinline|closed_segment| is
defined as:
% u::real. 0 $\leq$ u $\and$ u $\leq$ 1 
\begin{lstlisting}
  definition closed_segment :: 'a::real_vector $\Rightarrow$ 'a $\Rightarrow$ 'a set
  where closed_segment a b = {(1 - u) *$_R$ a + u *$_R$ b | u::real. 0 $\leq$ u $\land$ u $\leq$ 1 }
\end{lstlisting}
The fact that r lies strictly left 
of $\vec{p q}$ can be expressed differently by stating that ($p$, $q$ , $r$) 
are making a strictly counterclockwise turn. The three points are written as 
a tuple as it is again necessary to state the order of 
$p$, $q$ and $r$ when talking about a counterclockwise turn. In the following a counterclockwise turn will always refer to a strict 
counterclockwise turn. Checking if a point $r$ lies
strictly left of a vector is an operation that is essential for almost all
convex hull algorithms. To check if the points $((x_1,y_1)$ , $(x_2,y_2)$, $(x_3,y_3))$
make a counterclockwise turn, we can look at the sign of the determinant of the 
following matrix.

\[
\text{det} 
\begin{vmatrix}
1 & x_1 & y_1 \\
1 & x_2 & y_2 \\
1 & x_3 & y_3 \\
\end{vmatrix}
= x_1 (y_2 - y_3 ) + x_2 (y_3 - y_1)+ x_3  (y_1 - y_2)
\]
If the determinant is positive, we know that the sequence $(x_1,y_1), (x_2,y_2) , (x_3,y_3)$
makes a counterclockwise turn, if the determinant is zero we know that the three points are colinear 
and if the determinant is negative, we know the the sequence $(x_1,y_1), (x_2,y_2) , (x_3,y_3)$
makes a clockwise turn. %maybe include figure that show points making clockwise, counterclockwise turn
%and being colinear
In Isabelle the function that calculates the above determinant for three 
points is called \lstinline|det3|.
\begin{lstlisting}
  fun det3:: point $\Rightarrow$ point $\Rightarrow$ point $\Rightarrow$ real where 
  det3 (x1, y1) (x2, y2) (x3, y3) =
  x1 * y2 + y1 * x3 + x2 * y3 - y2 * x3 - y1 * x2 - x1 * y3"
\end{lstlisting}
Based on \lstinline|det3| the \lstinline|ccw'| predicate is defined, which expresses that 
three points (p,q,r) make a counterclockwise turn.
\begin{lstlisting}
  definition ccw' p q r $\longleftrightarrow$ 0 < det3 p q r
\end{lstlisting}
Lastly the predicate \lstinline|ccw'_seg p q r| holds if and only if r either lies 
counterclockwise of $\vec{pq}$ or r is contained in the closed segment between p and q.
\begin{lstlisting}
  definition ccw'_seg p q r = ccw' p q r $\lor$ r $\in$ closed_segment p q
\end{lstlisting}


\subsection{Order}
\label{sec:order}
In both algorithms we need to do the following operation. Given an corner $p$ of the convex polygon,
find another corner by searching for a point $q$ such that for all other points 
$r \in s$ either \lstinline|ccw' p q r| or \lstinline|r $\in$ closed_segment p q|
holds. In short, we search for a $q$ that fulfills  
\lstinline|$\forall$ r $\in$ s. ccw'_seg p q r|. 
Intuitively it makes sense that given a finite 
$s \subseteq \mathbb{R}^2$ and a corner of the convex hull polygon, 
we can find a unique next corner. Figuratively speaking, we rotate a line that 
starts in $p$ counterclockwise until we hit a point $q$, which is going to be the next
corner. If we hit several points at the same time, we are just going to take the point 
further away from $p$. Now to translate this into a formal framework, we start with
the previous definition of finding a $q$ that fulfills 
$\forall r \in s.$ \lstinline|(ccw'_seg p)| $q$ $r$. If \lstinline|(ccw'_seg p)| is a total
order on $s$, we know that such a $q$ exists. That's because q is the minimum
respect to the ordering \lstinline|(ccw'_seg p)|. For \lstinline|(ccw'_seg p)| to be a total order
and for later proofs it is necessary that we derive some form of transitivity for the
counterclockwise orientation. For example it should hold that if \lstinline|(ccw'_seg p a b)|
and \lstinline|(ccw'_seg p b c)| holds, then \lstinline|(ccw'_seg p a c)| should hold 
as well. The same implication should hold when using the \lstinline|(ccw'_seg p)| 
ordering instead of \lstinline|(ccw'_seg p)|. Altough straightforward, this kind of transitivity does not always
hold as the following examples shows. Clearly \lstinline|(ccw' p$_1$ p$_2$ p$_3$)| holds and also 
\lstinline|(ccw' p$_0$ p$_3$ p$_4$)|, but \lstinline|(ccw' p$_1$ p$_2$ p$_4$)| does not hold, instead 
\lstinline|(ccw' p$_1$ p$_4$ p$_2$)| holds. So in order for transitivity to hold, we need to restrict 
the set on which transitivity is supposed to hold. It can be shown that if there exists a $p_0$ such
that for all $r \in s$ it holds that \lstinline|ccw'_seg p$_0$ p$_1$ r| holds, then  
\lstinline|(ccw'_seg p$_1$)| is transitive on $s$. This restriction avoids the counterexample
for general transitivity from above. Transitivity also holds if there exists a point $p_0$ such 
that all $r \in s$ are lexicographically bigger than $p_0$, meaning $\forall r .$ \lstinline|lex p$_0$ r| holds,
where \lstinline|lex| is defined as.
\begin{lstlisting}
  definition lex:: point $\Rightarrow$ point $\Rightarrow$ bool where
  "lex p q $\longleftrightarrow$ (fst p < fst q $\lor$ fst p = fst q $\land$ snd p < snd q $\lor$ p = q)"
\end{lstlisting}
To check if $p$ is lexicographically smaller than $q$, we check if $p_x$ is smaller than $q_x$. If 
they are equal we check if $p_y \leq q_y$ holds. Now given for our reference set 
$ps \subseteq \mathbb{R}^2$ if $(\forall q \in ps.$  \lstinline| ccw'_seg  p_stl p_last q| $)\lor($
$\forall q \in ps.$ \lstinline|lex p_last q| $)$ holds, then the following lemmas can be proven.
\begin{lstlisting}
  lemma ccw'_seg_trans:
  assumes "p $\in$ ps" "q $\in$ ps" "k $\in$ ps"
  assumes "ccw'_seg p_last p q" "ccw'_seg p_last k p"
  shows "ccw'_seg p_last k q"
\end{lstlisting}
\begin{lstlisting}
  lemma ccw'_seg_total:
  assumes "p $\in$ ps" "q $\in$ ps"
  shows "ccw'_seg p_last p q $\lor$ ccw'_seg p_last q p"
\end{lstlisting}
\begin{lstlisting}
  lemma ccw'_seg_antisymmetric:
  assumes "ccw'_seg p_last p q $\land$ ccw'_seg p_last q p"
  shows "p = q"
\end{lstlisting}
Reflexivity directly follows from the definition of \lstinline|ccw'_seg|. Therefore we know
that there exists a unique $q$ such that $\forall r \in ps.$ \lstinline|(ccw'_seg p_last) q r|.
Notice how $ps$ was defined using $p\_last$.
\subsection{Convex Polygon}
\label{sec:conv_poly}
Both algorithms calculate the convex polygon that corresponds to the 
convex hull of the input set $s \subseteq \mathbb{R}^2$. This convex polygon 
is described by a list of points from $s$ that are the corners of this convex polygon.
So far, we just always just stated that the convex polygon corresponds to the 
convex hull, yet it is not obvious that this is the case. Therefore we require a  
description of a convex polygon in Isabelle/HOL and we need 
to know that this description is indeed equivalent to \lstinline|convex hull|, which is defined 
as Intersection of all convex sets that contain $s$. 
To be more precicse, we require a proof that the convex hull of the corners of such a 
convex polygon corresponds to the set of all points that lie within the polygon.
This fact was proven for a list of corners \lstinline|p0 # ps| that should
represent a convex polygon by Simon Hanssen.
\begin{lstlisting}
  lemma polygon_eq_convex_hull:
  assumes turns_only_left (p0 # ps)
      and sorted_wrt (ccw' p0) ps
      and 2 $\leq$ length ps
    shows list_all (encompasses p) (polychain_of (p0 # ps @ [p0]))
       $\longleftrightarrow$ p $\in$ convex hull (set (p0 # ps))"
\end{lstlisting}
To understand this proof, we need to first look at the definitions of all the predicates
used. First \lstinline|turns_only_left l| for a list l expresses that every three consecutive
points in the list are turning counterclockwise. This ensures that every interior angle of 
the polygon is less than or equal $180^\circ$, which is one of the typical definitions
of a convex polygon.
\begin{lstlisting}
  fun turns_only_left :: "point list $\Rightarrow$ bool" where
  "turns_only_left (p#q#r#ps) $\longleftrightarrow$ ccw' p q r $\land$ turns_only_left (q#r#ps)"|
  "turns_only_left _ = True"
\end{lstlisting}
Next \lstinline|sorted_wrt (ccw' p0)|, where p0 is the start or our list of corners, states that
for every corner $p$ in the list, all corners that are behind it in the list, lie counterclockwise
of $\overrightarrow{p0  p}$.
\begin{lstlisting}
  fun sorted_wrt :: "('a $\Rightarrow$ 'a $\Rightarrow$ bool) $\Rightarrow$ 'a list $\Rightarrow$ bool" where
"sorted_wrt P [] = True" |
"sorted_wrt P (x # ys) = (($\forall$ y $\in$ set ys. P x y) $\land$ sorted_wrt P ys)"
\end{lstlisting}     
This property avoids degenerations as shown in Figure x.
Lastly \lstinline| 2 $\leq$ length ps | is needed, as the definition does not work in the case 
of two corners, where the polygon is just a closed segment between two points.
Now given a list \lstinline|p0 # ps| fulfills these properties, then we know that this list
describes a list of corners of a convex polygon and the following statement holds.
\begin{lstlisting}
  list_all (encompasses p) (polychain_of (p0 # ps @ [p0]))
       $\longleftrightarrow$ p $\in$ convex hull (set (p0 # ps))
\end{lstlisting}  
Where \lstinline|polychain_of (p0 # ps @ [p0])| is just the list of all tuples of two 
consecutive points in the list and 
\lstinline|encompasses p seg = det3 (fst seg) (snd seg) p $\geq$ 0| states that p lies
counterclockwise (or colinear)  of the vector  $\overrightarrow{(fst \; seg) (snd \; seg)}$. 
With $(fst \; seg)$ being the first point in the tuple  
\lstinline|seg| and $(snd \; seg)$ being the second point in the tuple \lstinline|seg|.
\begin{lstlisting}
  fun polychain_of where
  "polychain_of [] = []"
| "polychain_of [p2] = []"
| "polychain_of (p1#p2#ps) = (p1, p2) # polychain_of (p2 # ps)"
\end{lstlisting}  
The \lstinline|list_all P l| predicate states that the condition \lstinline|P| has to hold
for every element in the list \lstinline|l|.
Consequently \lstinline|list_all (encompasses p) (polychain_of (p0 # ps @ [p0]))| states that p lies
inside the polygon defined by \lstinline|p0 # ps| as it requires that p lies counterclockwise (or colinear) 
of every edge of the polygon. Therefore the lemma \lstinline|polygon_eq_convex_hull| states that a point p
lies inside the convex polygon defined by \lstinline|p0 # ps| if and only if p is in the convex hull
of \lstinline|set (p0 # ps)|. Now we have the definition of a convex polygon and the proof that
the convex hull of the corners of such a 
polygon corresponds to the set of all points that lie within the polygon. Based on this we can show 
that the inspected algorithms, for an input set $s$, compute a convex polygon according to the definition  
and that the convex hull that corresponds to this convex polygon is indeed the convex hull of $s$.




% So given for a list of points  \lstinline|p0 # ps| that contains at least three points, \lstinline|turns_only_left (p0#ps)|
% and \lstinline|sorted_wrt (ccw' p0) ps| holds, then we know that 
% Then we will need to show, that all other points in $s \subseteq \mathbb{R}^2$ (not corners)
% lie inside the convex polygon that is spanned by the corners. 

% Therefore we need a formalization for the convex polygon is and a proof that the
% convex hull of the corners of the polygon is the set of points that lie inside the polygon.  
% From this we will get that both algorithms calculate a list . Then we need to show that all other points in the input set $s$
% lie within this convex polygon and from that we get that the algorithms indeed calculate 
% the convex hull of the input set. In Isabelle/HOL, the 



%might be better to define with {p,q} \in S^2 (i.e. not ordered)



%But the convex hull can also be defined 
%as the set of all convex combinations of points in S, which can be proven equivalent to the previous definition.   


%how is it done in Isabelle
\section{Jarvis-March Algorithm}
\subsection{Definition of the Algorithm}
\label{jarvis:def}
The Jarvis March or Gift-Wrapping Algorithm is a simple output-sensitive way of calculating
the convex hull of a given finite set $S \subseteq \mathbb{R}^2$ of points. It calculates the
convex hull by calculating the corresponding convex polygon and returning
an ordered list of the corners of the polygon. The algorithm has runtime O(n * h), where n 
is the number of points in S and h is the number of points that lie on the convex hull or
the number of corners on the calculated polygon to be more precicse. 
The algorithm starts by choosing a point that is guaranteed to lie on the convex hull, 
for example $p_0 = min_y min_x S$ (the lexicographical minimum). Then the next corner of the convex polygon
is found by searching a $p_1$ such that every point $r \in s$ lies counterclockwise
of $\vec{p_0 p_1}$ or is contained in the closed segment between $p_0$ and $p_1$, meaning
$\forall r \in ps.$ \lstinline|(ccw'_seg p$_0$) p$_1$ r| should hold. 
As explained in \ref{sec:order} we know  that such a $p_1$ exists, because 
$\forall r \in ps.$ \lstinline|lex p$_0$ r| holds. In Isabelle the definition for
finding the minimum with respect to the total order \lstinline|(ccw'_seg p$_0$)| is.
\begin{lstlisting}
  definition ccw'_seg_min :: " point set $\Rightarrow$ point" where
  "ccw'_seg_min ps =  (THE p. p $\in$ ps $\land$ ($\forall$ q $\in$ ps. ccw'_seg p0 p q))"
\end{lstlisting}  
Now from \ref{sec:basics}, we 
know that $(p_0,p_1)$ is an edge of the wanted convex polygon and 
we know that $p_1$ is once again a point on
the convex hull and a corner of the polygon as $(p_0,p_1)$ fulfills $\forall r \in ps.$ \lstinline|(ccw'_seg p$_0$) p$_1$ r|. 
Therefore we can repeat the previous step
and search for a $p_2$ that fulfills $\forall r \in ps.$ \lstinline|(ccw'_seg p$_1$) p$_2$ r|. 
Once again according to \ref{sec:order}, we know that $\forall r \in ps.$ \lstinline|(ccw'_seg p$_0$) p$_1$ r|
holds and therefore \lstinline|(ccw'_seg p$_1$)| is a total order and a unique $p_2$ exists.
Again $p_2$ has to be a corner of the convex polygon and $(p_1,p_2)$ an edge on of the
polygon. The algorithm continues until a $p_h = p_0$ is found to be the next point and
stops, because the first corner of the polygon is encountered again. 
The ordered sequence of points $p_0,p_q, ... , p_{h-1}$ are the corners of the convex polygon 
and $(p_0,p_1),(p_1,p_2) ... ,(p_{h-2},p_{h-1}), (p_{h-1},p_0)$ are the edges of the polygon.
This repeated finding of the next corner is defined as the function \lstinline|wrap|, where
$q$ is the last minimum that was found and ps is the current set of points we want to find
the convex hull of.
\begin{lstlisting}
function wrap :: "point $\Rightarrow$ point set $\Rightarrow$ point list" where
"wrap q ps =  
(if q = p0 then [] else q#(wrap (ccw'_seg_min q ps) (ps - {q})  ) )"
\end{lstlisting}
The last minimum $q$ is prepended to the list of corners we will return, if we not yet arrived
at the first corner $p0$ again. The next corners are found by  recursively calling \lstinline|wrap|
with the next corner or minimum \lstinline|ccw'_seg_min q ps| and the set $ps - \{q\}$. $q$ can be
removed from the set of points we search for the next corner, as $q$ can not be a corner
of the polygon again.
Lastly the algorithm Javis March is defined by an inital call to wrap, but $p0$ is this time
not removed from the set $ps$ we search for the next corner, because $p0$ is the only corner we
can and must encounter twice.
\begin{lstlisting}
  definition "jarvis_march = to_set (wrap (ccw'_seg_min p0 ps) ps)"
\end{lstlisting}  
The \lstinline|to_set| function just turns the list or corners into the appropriate 
definition of the set of points that lie inside the polygon (see \ref{sec:conv_poly}).  
\begin{lstlisting}
  fun to_set :: "point list $\Rightarrow$ point set" where
"to_set [] = {p0}" |
"to_set [p] = closed_segment p0 p" |
"to_set qs = {p. list_all (encompasses p) (polychain_of (p0#qs@[p0]))}" 
\end{lstlisting}
The special cases of \lstinline|wrap| returning an empty list or an list with only one element
need more explaination. If \lstinline|(wrap (ccw'_seg_min p0 ps) ps) = []|, then we know
\lstinline|ccw'_seg_min p0 ps = p0| has to hold and therefore $\forall r \in ps.$ \lstinline|ccw'_seg p$_0$ p$_0$ r|.
Intuitively it should be clear, that the only point $r$ that fulfills \lstinline|ccw'_seg p$_0$ p$_0$ r| is $p0$ itself
and therefore $ps$ has to only contain $p0$ and the convex hull of a single point is a set containing this
very point. If \lstinline|(wrap (ccw'_seg_min p0 ps) ps) = [p]|, then we know 
$\forall r \in ps.$ \lstinline|ccw'_seg p$_0$ p r| and $\forall r \in ps.$ \lstinline|ccw'_seg p p$_0$ r|.
Again from geometric intuition it should be clear that  $\forall r \in ps.$ \lstinline|r $\in$ closed_segment p0 p|
should hold, as \lstinline|ccw' p$_0$ p r| or \lstinline|ccw' p p$_0$ r| instantly leads to a contradiction.
The last case of the \lstinline|to_set| function just applies the definition for the 
set of points inside inside a convex polygon, as introduced in \ref*{sec:conv_poly}. 
\subsection{Jarvis March calculates the Convex Hull}
In the following let $ps \subseteq \mathbb{R}^2$ be the finite set of points of which we want to calculate
the convex hull and let $p0 = min_y min_x ps$ be the lexicographical minimum with which we start 
Jarvis March, i.e. our first corner of the convex polygon. In Isabelle terms, we assume 
\lstinline|$\forall$p $\in$  ps. lex p0 p| \space , \space  \lstinline|p0 $\in$ ps| \space  and \space  \lstinline|finite ps|.
First we need to show that the recursive wrap function terminates. 
\begin{lstlisting}
  lemma wrap_dom:
  assumes q $\in$ qs $\land$ p0 $\in$ qs
  assumes "qs $\subseteq$ ps"
  assumes "q = p0 $\lor$ ($\forall$q' $\in$ qs. ccw'_seg p_stl q q')"
  shows "wrap_dom (q,qs)"
\end{lstlisting}
This lemma follows from the step by step 
description of \ref*{jarvis:def}. In every step our last minimum $q$ was either equal to $p0$ (in the beginning) 
which fulfills $\forall r \in ps.$ \lstinline|lex p$_0$ r| or our last minimum fulfilled 
$\forall r \in ps.$ \lstinline|ccw'_seg p q r| (found with \lstinline|wrap|) for some $p$.
In both cases \lstinline|(ccw'_seg q)| is a total order and a new minimum $q_{next}$ such that 
$\forall r \in ps.$ \lstinline|ccw'_seg q q_{next} r| holds, exists (see \ref*{sec:order}). So 
\lstinline| ccw'_seg_min q qs| and therefore every recursive call to \lstinline|wrap| is well-defined. 
Additionally the size of the set with which \lstinline|wrap| is recursively called 
decreases in every iteration. Hence the call  \lstinline|(wrap (ccw'_seg_min p0 ps) ps)|
will terminate.
Now we need to show that the list that \lstinline|(wrap (ccw'_seg_min p0 ps) ps)| 
returns represents a correct convex polygon.
\begin{lstlisting}
lemma wrap_sorted_ccw':
  shows "sorted_wrt (ccw' p0) (wrap (ccw'_seg_min p0 ps) ps)"

lemma wrap_turns_left:
  shows "turns_only_left (wrap (ccw'_seg_min p0 ps) ps)"
\end{lstlisting}
To do this, we first show that the inner call \lstinline|wrap (ccw'_seg_min q qs) (qs - {q})|,
where we assume that $\forall r \in  qs.$ \lstinline|ccw'_seg p q r| holds
for the last minimum $q$ and some $p$, produces a list that is \lstinline|sorted_wrt (ccw' p0)|.
\begin{lstlisting}
  lemma wrap_sorted:
  assumes wrap (ccw'_seg_min q qs) (qs - {q}) = ls
  assumes q $\in$ qs  $\land$ p0 $\in$ qs
  assumes qs $\subseteq$ ps
  assumes ($\forall$r $\in$  qs. ccw'_seg p q r) $\land$ (p0 $\neq$ q)
  shows sorted_wrt (ccw' p0) ls
\end{lstlisting}



The algorithm is simpler than the Graham Scan or the Chan's algorithm and has a worse
runtime than both unless h is small. Graham Scan achieves a $O(n log(n))$ runtime and 
Chan's algorithm a $O(n log(h))$ runtime. If h is small Jarvis March 
can be faster than Graham Scan.


\section{Graham Scan}

\section{Chans Algorithm}


Citation test~\parencite{latex}.

Acronyms must be added in \texttt{main.tex} and are referenced using macros. The first occurrence is automatically replaced with the long version of the acronym, while all subsequent usages use the abbreviation.

E.g. \texttt{\textbackslash ac\{TUM\}, \textbackslash ac\{TUM\}} $\Rightarrow$ \ac{TUM}, \ac{TUM}

For more details, see the documentation of the \texttt{acronym} package\footnote{\url{https://ctan.org/pkg/acronym}}.
\subsection{Subsection}

See~\autoref{tab:sample}, \autoref{fig:sample-drawing}, \autoref{fig:sample-plot}, \autoref{fig:sample-listing}.

\begin{table}[htpb]
  \caption[Example table]{An example for a simple table.}\label{tab:sample}
  \centering
  \begin{tabular}{l l l l}
    \toprule
      A & B & C & D \\
    \midrule
      1 & 2 & 1 & 2 \\
      2 & 3 & 2 & 3 \\
    \bottomrule
  \end{tabular}
\end{table}

\begin{figure}[htpb]
  \centering
  % This should probably go into a file in figures/
  \begin{tikzpicture}[node distance=3cm]
    \node (R0) {$R_1$};
    \node (R1) [right of=R0] {$R_2$};
    \node (R2) [below of=R1] {$R_4$};
    \node (R3) [below of=R0] {$R_3$};
    \node (R4) [right of=R1] {$R_5$};

    \path[every node]
      (R0) edge (R1)
      (R0) edge (R3)
      (R3) edge (R2)
      (R2) edge (R1)
      (R1) edge (R4);
  \end{tikzpicture}
  \caption[Example drawing]{An example for a simple drawing.}\label{fig:sample-drawing}
\end{figure}

\begin{figure}[htpb]
  \centering

  \pgfplotstableset{col sep=&, row sep=\\}
  % This should probably go into a file in data/
  \pgfplotstableread{
    a & b    \\
    1 & 1000 \\
    2 & 1500 \\
    3 & 1600 \\
  }\exampleA
  \pgfplotstableread{
    a & b    \\
    1 & 1200 \\
    2 & 800 \\
    3 & 1400 \\
  }\exampleB
  % This should probably go into a file in figures/
  \begin{tikzpicture}
    \begin{axis}[
        ymin=0,
        legend style={legend pos=south east},
        grid,
        thick,
        ylabel=Y,
        xlabel=X
      ]
      \addplot table[x=a, y=b]{\exampleA};
      \addlegendentry{Example A}
      \addplot table[x=a, y=b]{\exampleB};
      \addlegendentry{Example B}
    \end{axis}
  \end{tikzpicture}
  \caption[Example plot]{An example for a simple plot.}\label{fig:sample-plot}
\end{figure}
