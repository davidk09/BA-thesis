!TeX root = ../main.tex
% Add the above to each chapter to make compiling the PDF easier in some editors.

\chapter{Definitions and Algorithms}\label{chapter:definitions_algorithms}

\section{Convex Hull}
\label{sec:conv_hull}
First the Convex Hull will be defined. A set $ s \subseteq \mathbb{R}^2 $ is convex if for  every
two points p and q in $s$ it holds that all points on the line segment connecting p and q 
are in $s$ again. This can be expressed, as the fact that the any convex combination of 
p and q has to be in $s$ again. In Isabelle the convex predicate is defined exactly this way:
\begin{lstlisting}
  definition convex :: 'a real_vector set $\Rightarrow$ bool where
  convex s $\longleftrightarrow$ ($\forall$x$\in$s. $\forall$y$\in$s. $\forall$u$\geq$0. $\forall$v$\geq$s. u + v = 1 $\longrightarrow$ u *$_R$ x + v *$_R$ y $\in$ s)
\end{lstlisting}
The convex hull of a set $s$ is the smallest convex set in which $s$ is contained. 
There are several alternative ways in which the convex hull can be defined.
One possible way is to define it as the intersection of all convex sets containing $s$, which 
is also the definition used in Isabelle/HOL. We have already seen the convex predicate, the hull
predicate is defined as the intersection of all sets t that contain s and fulfill the predicate S.
\begin{lstlisting}
    definition hull :: (a' set $\Rightarrow$ bool) $\Rightarrow$ a' set $\Rightarrow$ a' set where
    S hull s = $\bigcap${t. S t $\land$ s $\subseteq$ t}
\end{lstlisting}
Consequently \lstinline|convex hull s| refers to the intersection of all convex sets that 
contain $s$ and therefore the convex hull of the set $ s \subseteq \mathbb{R}^2 $.
In the two dimensional case for a finite $s \subset \mathbb{R}^2$, the convex hull CH of $s$ 
is a convex polygon and all the corners of this convex polygon are points from S (see figure 1). \cite{de2000computational}
The edges $E \subseteq s^2$  of the polygon are exactly those $\{p,q\} \in s^2$ for which 
either all points in $s$ lie left of the line $\overline{pq}$ or all points in $s$ lie left of the 
line $\overline{qp}$
An edge connecting two points $(p,q) \in S^2$  is an edge of the convex polygon iff. all points 
lie to the left of the line $\overline{pq}$ connecting p and q. Similarly the set of all 
edges of the convex polygon can be defined as all $(p,q) \in S^2$ for which
all points in S lie to the right of $\overline{pq}$  
%might be better to define with {p,q} \in S^2 (i.e. not ordered)
As this thesis will focus on the two dimensional case and only give an outlook on the the 
three dimensional case, the examined algorithms compute a convex polygon for a given $S \subset \mathbb{R}^2$. 



%But the convex hull can also be defined 
%as the set of all convex combinations of points in S, which can be proven equivalent to the previous definition.   


%how is it done in Isabelle
\section{Jarvis-March Algorithm}
The Jarvis March or Gift-Wrapping Algorithm is a simple output-sensitive way of calculating
the convex hull of a given finite set $S \subseteq \mathbb{R}^2$ of points. It calculates the
convex hull by calculating the corresponding convex polygon and returning
an ordered list of the corners of the polygon. The algorithm has runtime O(n * h), where n 
is the number of points in S and h is the number of points that lie on the convex hull or
the number of corners on the calculated polygon to be more precicse. 
First we will assume that no three points in S are colinear.
The algorithm starts by choosing a point that is guaranteed to lie on the convex hull, 
for example a $p_0 = min_y min_x S$. Then the next corner of the convex polygon
is found by searching a $p_1$ such that all points in S lie to the left of the 
line $\overline{p_0 p_1}$. As explained in \ref{sec:conv_hull} we know that $(p_0,p_1)$ is 
an edge of the wanted convex polygon and we know that q is once again a point on
the conex hull, i.e. a corner of the polygon. Therefore we can repeat the previous step
and search for a $p_2$ such that all points in S lie left to the line $\overline{p_1 p_2}$.
Again $p_2$ has to be a corner of the convex polygon and $(p_1,p_2)$ an edge on of the
polygon. The algorithm continues until a $p_h = p_0$ is found to be the next point and
stops, because the first corner of the polygon is encountered again. 
The ordered sequence of points $p_0,p_q, ... , p_{h-1}$ are the corners of the convex polygon 
and $(p_0,p_1),(p_1,p_2) ... ,(p_{h-2},p_{h-1}), (p_{h-1},p_0)$ are the edges of the polygon.
Now without the assumption that no three points are colinear, we require more rigorous definitions.
Given a $p_i$ that is a corner of the convex polygon the next corner $p_{i+1}$ 
has to fulfill the following condition for all $q \in S$.
Either q lies strictly left of $\overline{p_i p_{i+1}}$ ($p_i$
, $p_{i+1}$ and $q$ are not colinear) or $q$ is contained in the closed
segment between $p_i$ and $p_{i+1}$. In the following  a point $q$ lying strictly
left of a line $\overline{p_i p_{i+1}}$ will be expressed as $q$ lying
counterclockwise of the line $\overline{p_i p_{i+1}}$. This clarification avoids, that
points which are not a corner but still lie on the convex hull are ignored (see figure 2).    
The algorithm is simpler than the Graham Scan or the Chan's algorithm and has a worse
runtime than both unless h is small. Graham Scan achieves a $O(n log(n))$ runtime and 
Chan's algorithm a $O(n log(h))$ runtime. If h is small Jarvis March 
can be faster than Graham Scan.
\begin{lstlisting}[language=isabelle]
  lemma turns_only_right st $\Longrightarrow$
  turns_only_right (grahamsmarch qs st)
\end{lstlisting}
  

\section{Graham Scan}

\section{Chans Algorithm}


Citation test~\parencite{latex}.

Acronyms must be added in \texttt{main.tex} and are referenced using macros. The first occurrence is automatically replaced with the long version of the acronym, while all subsequent usages use the abbreviation.

E.g. \texttt{\textbackslash ac\{TUM\}, \textbackslash ac\{TUM\}} $\Rightarrow$ \ac{TUM}, \ac{TUM}

For more details, see the documentation of the \texttt{acronym} package\footnote{\url{https://ctan.org/pkg/acronym}}.
\subsection{Subsection}

See~\autoref{tab:sample}, \autoref{fig:sample-drawing}, \autoref{fig:sample-plot}, \autoref{fig:sample-listing}.

\begin{table}[htpb]
  \caption[Example table]{An example for a simple table.}\label{tab:sample}
  \centering
  \begin{tabular}{l l l l}
    \toprule
      A & B & C & D \\
    \midrule
      1 & 2 & 1 & 2 \\
      2 & 3 & 2 & 3 \\
    \bottomrule
  \end{tabular}
\end{table}

\begin{figure}[htpb]
  \centering
  % This should probably go into a file in figures/
  \begin{tikzpicture}[node distance=3cm]
    \node (R0) {$R_1$};
    \node (R1) [right of=R0] {$R_2$};
    \node (R2) [below of=R1] {$R_4$};
    \node (R3) [below of=R0] {$R_3$};
    \node (R4) [right of=R1] {$R_5$};

    \path[every node]
      (R0) edge (R1)
      (R0) edge (R3)
      (R3) edge (R2)
      (R2) edge (R1)
      (R1) edge (R4);
  \end{tikzpicture}
  \caption[Example drawing]{An example for a simple drawing.}\label{fig:sample-drawing}
\end{figure}

\begin{figure}[htpb]
  \centering

  \pgfplotstableset{col sep=&, row sep=\\}
  % This should probably go into a file in data/
  \pgfplotstableread{
    a & b    \\
    1 & 1000 \\
    2 & 1500 \\
    3 & 1600 \\
  }\exampleA
  \pgfplotstableread{
    a & b    \\
    1 & 1200 \\
    2 & 800 \\
    3 & 1400 \\
  }\exampleB
  % This should probably go into a file in figures/
  \begin{tikzpicture}
    \begin{axis}[
        ymin=0,
        legend style={legend pos=south east},
        grid,
        thick,
        ylabel=Y,
        xlabel=X
      ]
      \addplot table[x=a, y=b]{\exampleA};
      \addlegendentry{Example A}
      \addplot table[x=a, y=b]{\exampleB};
      \addlegendentry{Example B}
    \end{axis}
  \end{tikzpicture}
  \caption[Example plot]{An example for a simple plot.}\label{fig:sample-plot}
\end{figure}
